%%=============================================================================
%% Inleiding
%%=============================================================================

\chapter{\IfLanguageName{dutch}{Inleiding}{Introduction}}%
\label{ch:inleiding}

Dit bachelorproefonderzoek behandelt de problematiek rond CA Gen-gebaseerde legacy-applicaties. CA Gen is een verouderde technologie waarvan de expertise steeds meer afneemt. Dit zorgt ervoor dat het onderhouden en optimaliseren van deze code moeilijker word. Het onderzoek heeft als doel een high-level framework te ontwikkelen die developpers hulp moet bieden tijdens het migreren van zo een applicatie naar bijvoorbeeld PL/I.

\section{\IfLanguageName{dutch}{Probleemstelling}{Problem Statement}}%
\label{sec:probleemstelling}

Dit onderzoek speelt zich af in de context van het bedrijf Arcelor Mittal, daar draaien nog veel CA Gen-gebaseerde applicaties. Het framework moet developpers in de toekomst helpen om deze applicaties systematisch om te zetten naar PL/I.

\section{\IfLanguageName{dutch}{Onderzoeksvraag}{Research question}}%
\label{sec:onderzoeksvraag}

De centrale onderzoeksvraag luidt als volgt: Hoe kan een high-level framework worden opgesteld om CA Gen-applicaties succesvol te migreren naar PL/I code? Maar belangrijke deelvragen zijn bevoorbeeld: “Welke delen van de pseudocode kan je rechtstreeks vertalen?”.

\section{\IfLanguageName{dutch}{Onderzoeksdoelstelling}{Research objective}}%
\label{sec:onderzoeksdoelstelling}

Het doel van dit onderzoek is een high-level framework te maken die een hulplijn moeten bieden tijdens het migratieproces. Dit zal bewezen worden met een proof-of-concept.

\section{\IfLanguageName{dutch}{Opzet van deze bachelorproef}{Structure of this bachelor thesis}}%
\label{sec:opzet-bachelorproef}

% Het is gebruikelijk aan het einde van de inleiding een overzicht te
% geven van de opbouw van de rest van de tekst. Deze sectie bevat al een aanzet
% die je kan aanvullen/aanpassen in functie van je eigen tekst.

De rest van deze bachelorproef is als volgt opgebouwd:

In Hoofdstuk~\ref{ch:stand-van-zaken} wordt een overzicht gegeven van de stand van zaken binnen het onderzoeksdomein, op basis van een literatuurstudie.

In Hoofdstuk~\ref{ch:methodologie} wordt de methodologie toegelicht en worden de gebruikte onderzoekstechnieken besproken om een antwoord te kunnen formuleren op de onderzoeksvragen.

% TODO: Vul hier aan voor je eigen hoofstukken, één of twee zinnen per hoofdstuk

In Hoofdstuk~\ref{ch:conclusie}, tenslotte, wordt de conclusie gegeven en een antwoord geformuleerd op de onderzoeksvragen. Daarbij wordt ook een aanzet gegeven voor toekomstig onderzoek binnen dit domein.