%==============================================================================
% Sjabloon onderzoeksvoorstel bachproef
%==============================================================================
% Gebaseerd op document class `hogent-article'
% zie <https://github.com/HoGentTIN/latex-hogent-article>

% Voor een voorstel in het Engels: voeg de documentclass-optie [english] toe.
% Let op: kan enkel na toestemming van de bachelorproefcoördinator!
\documentclass{hogent-article}
\usepackage{color,soul}

% Invoegen bibliografiebestand
\addbibresource{voorstel.bib}

% Informatie over de opleiding, het vak en soort opdracht
\studyprogramme{Professionele bachelor toegepaste informatica}
\course{Bachelorproef}
\assignmenttype{Onderzoeksvoorstel}
% Voor een voorstel in het Engels, haal de volgende 3 regels uit commentaar
% \studyprogramme{Bachelor of applied information technology}
% \course{Bachelor thesis}
% \assignmenttype{Research proposal}

\academicyear{2025-2026} % TODO: pas het academiejaar aan

% TODO: Werktitel
\title{Hoe kan een high-level framework worden opgesteld om CA Gen-applicaties succesvol te migreren naar PL/I code?}

% TODO: Studentnaam en emailadres invullen
\author{Loïc Vandecandelaere}
\email{loic.Vandecandelaere@student.hogent.be}
\projectrepo{https://github.com/loicvdc/Bachelorproef-Lo-c-Vandecandelaere.git}
% TODO: Medestudent
% Gaat het om een bachelorproef in samenwerking met een student in een andere
% opleiding? Geef dan de naam en emailadres hier
% \author{Yasmine Alaoui (naam opleiding)}
% \email{yasmine.alaoui@student.hogent.be}

% TODO: Geef de co-promotor op
\supervisor[Co-promotor]{ (, \href{mailto:}{})}

% Binnen welke specialisatierichting uit 3TI situeert dit onderzoek zich?
% Kies uit deze lijst:
%
% - Mobile \& Enterprise development
% - AI \& Data Engineering
% - Functional \& Business Analysis
% - System \& Network Administrator
% - Mainframe Expert
% - Als het onderzoek niet past binnen een van deze domeinen specifieer je deze
%   zelf
%
\specialisation{Mainframe Expert}
\keywords{Framework, CA Gen, PL/I, Migratie}

\begin{document}

\begin{abstract}
  Dit bachelorproefonderzoek behandelt de problematiek van Ca Gen-gebaseerde legacy-applicaties, waarvan de beperkte beschikbaarheid van expertise en verouderde technologie de onderhoudbaarheid en toekomstbestendigheid onder druk zetten. Het onderzoek heeft als doel een generiek en high-level framework te ontwikkelen voor de systematische analyse, reverse engineering en migratie van Ca Gen-applicaties naar PL/I. De centrale onderzoeksvraag focust op de mate waarin dit framework een correcte en efficiënte migratie kan ondersteunen. Het framework begeleidt ontwikkelaars doorheen het volledige migratieproces en biedt ondersteuning via best practices en een toelichting van veelvoorkomende valkuilen. Het framework wordt uitgewerkt aan de hand van overzichtelijke tekst en figuren, met als doel een duidelijke en vlot leesbare leidraad te vormen. 
\end{abstract}

\tableofcontents

% De hoofdtekst van het voorstel zit in een apart bestand, zodat het makkelijk
% kan opgenomen worden in de bijlagen van de bachelorproef zelf.
%---------- Inleiding ---------------------------------------------------------

% TODO: Is dit voorstel gebaseerd op een paper van Research Methods die je
% vorig jaar hebt ingediend? Heb je daarbij eventueel samengewerkt met een
% andere student?
% Zo ja, haal dan de tekst hieronder uit commentaar en pas aan.

%\paragraph{Opmerking}

% Dit voorstel is gebaseerd op het onderzoeksvoorstel dat werd geschreven in het
% kader van het vak Research Methods dat ik (vorig/dit) academiejaar heb
% uitgewerkt (met medesturent VOORNAAM NAAM als mede-auteur).
% 

\section{Inleiding}%
\label{sec:inleiding}

Veel organisaties maken vandaag nog gebruik van legacyapplicaties die ontwikkeld zijn met CA Gen, een modelgedreven ontwikkeltool die in het verleden werd ingezet voor uiteenlopende toepassingen. Hoewel deze applicaties doorgaans stabiel functioneren, vormt de verouderde technologie, in combinatie met een sterk afnemende beschikbaarheid van CA Gen-expertise, een belangrijk knelpunt voor onderhoud en verdere evolutie. Binnen de context van deze bachelorproef vormt een concrete CA Gen-applicatie binnen een industriële bedrijfsomgeving (ArcelorMittal) de uitgangssituatie voor het onderzoek.

De doelgroep van deze bachelorproef bestaat uit IT-professionals binnen deze organisatie, in het bijzonder applicatiebeheerders, softwareontwikkelaars en technische architecten die instaan voor het onderhoud en de modernisering van CA Gen-gebaseerde systemen.

De centrale probleemstelling in deze casus is dat de huidige CA Gen-applicatie sterk afhankelijk is van specifieke tooling en schaarse kennis, wat leidt tot verhoogde onderhoudsrisico’s en een beperkte flexibiliteit bij aanpassingen of optimalisaties. Daarnaast ontbreekt een gestructureerde en herbruikbare aanpak om dergelijke applicaties op een gecontroleerde manier te analyseren en te migreren.

Vanuit deze probleemstelling wordt de volgende centrale onderzoeksvraag geformuleerd:

Hoe kan een high-level framework worden opgesteld om CA Gen-applicaties succesvol te migreren naar PL/I-code?

Om deze centrale onderzoeksvraag te beantwoorden, worden onderstaande deelvragen onderzocht.

Deelvragen binnen het probleemdomein

Wat zijn de belangrijkste kenmerken en architecturale eigenschappen van CA Gen-applicaties?

Welke functionaliteiten, afhankelijkheden en technische componenten zijn typisch aanwezig binnen een CA Gen-applicatie?

Welke beperkingen en valkuilen brengt het gebruik van CA Gen met zich mee in het kader van onderhoud en modernisering?

Deelvragen binnen het oplossingsdomein

Welke stappen zijn noodzakelijk om een CA Gen-applicatie op een gestructureerde manier te analyseren en voor te bereiden op migratie?

Hoe kan CA Gen-functionaliteit systematisch worden vertaald naar PL/I-code?

Welke richtlijnen en best practices kunnen ontwikkelaars ondersteunen tijdens het migratieproces?

In welke mate is het voorgestelde migratieframework herbruikbaar voor andere programmeertalen dan PL/I?

Het doel van deze bachelorproef is het opstellen van een duidelijk en praktisch stappenplan dat ontwikkelaars binnen een organisatie kunnen gebruiken om CA Gen-applicaties te migreren naar PL/I-code. Een secundair doel is het evalueren van de toepasbaarheid van dit stappenplan voor andere programmeertalen. Het onderzoek wordt als geslaagd beschouwd wanneer het stappenplan volledig, duidelijk en praktisch toepasbaar is, alle migratiestappen beschrijft en ontwikkelaars effectief begeleidt doorheen het volledige migratietraject.

%---------- Stand van zaken ---------------------------------------------------

\section{Literatuurstudie}%
\label{sec:literatuurstudie}

Binnen het onderzoeksdomein van CA Gen is de beschikbare academische en professionele literatuur beperkt. Dit is voornamelijk te verklaren door de ouderdom van de technologie en het afgenomen gebruik ervan in moderne softwareontwikkeling. Specifiek onderzoek naar de migratie van CA Gen-applicaties is schaars, wat de noodzaak onderstreept van praktijkgericht en toegepast onderzoek binnen dit domein. De belangrijkste formele bron voor dit onderzoek is de officiële documentatie van Broadcom, met name CA Gen Tutorial (Release 8.5) \textcite{Broadcom2013}. Deze documentatie biedt inzicht in de architectuur, ontwikkelprincipes en functionele opbouw van CA Gen-applicaties en vormt daarmee de basis voor de analyse en reverse engineering binnen deze bachelorproef. Daarnaast wordt gebruikgemaakt van interne documentatie die ter beschikking wordt gesteld door ArcelorMittal, waarin de specifieke implementatie en context van CA Gen binnen de organisatie wordt toegelicht. Deze combinatie van officiële productdocumentatie en bedrijfsspecifieke informatie maakt het mogelijk om de technologie correct te interpreteren binnen de gekozen casus.

Aangezien literatuur specifiek gericht op CA Gen-migraties beperkt is, wordt dit onderzoek verder ondersteund door bestaande studies en publicaties rond legacy-systemen en softwaremodernisering in bredere zin. Verschillende auteurs benadrukken dat verouderde systemen een aanzienlijke uitdaging vormen op het vlak van onderhoudbaarheid, veiligheid en integratie met moderne technologieën \textcite{StrategicMigration2024}. Daarnaast tonen praktijkgerichte publicaties aan dat organisaties steeds vaker geconfronteerd worden met kennisverlies en technologische afhankelijkheid bij 

legacy-platformen, wat migratie of modernisering noodzakelijk maakt \textcite{Hartholt2024}. Deze literatuur biedt waardevolle inzichten in algemene migratiestrategieën, evaluatiecriteria en risico’s, die als inspiratie en theoretisch kader dienen voor het opstellen van het voorgestelde migratieframework. Door deze algemene kennis toe te passen op de specifieke context van CA Gen, wordt een brug geslagen tussen bestaande migratieonderzoeken en de concrete probleemstelling van deze bachelorproef.

% Voor literatuurverwijzingen zijn er twee belangrijke commando's:
% \autocite{KEY} => (Auteur, jaartal) Gebruik dit als de naam van de auteur
%   geen onderdeel is van de zin.
% \textcite{KEY} => Auteur (jaartal)  Gebruik dit als de auteursnaam wel een
%   functie heeft in de zin (bv. ``Uit onderzoek door Doll & Hill (1954) bleek
%   ...'')


%---------- Methodologie ------------------------------------------------------
\section{Methodologie}%
\label{sec:methodologie}

Het onderzoek vertrekt vanuit een bestaande CA Gen-applicatie als casus, binnen Arcelor Mittal, die fungeert als representatief voorbeeld voor de problematiek van CA Gen-gebaseerde legacy-systemen.

\subsection{Probleemanalyse en contextstudie}

In een eerste fase wordt de huidige probleemsituatie in kaart gebracht door middel van documentanalyse en overleg met betrokken \newline IT-professionals binnen de organisatie. Welke rol speelt CA Gen binnen het applicatielandschap? Welke functionele en architecturale componenten bevat de CA Gen-applicatie? Zijn er database- of 
API-afhankelijkheden? Kan de database van CA Gen vervangen worden door DB2? De officiële CA Gen-documentatie van Broadcom en interne bedrijfsdocumentatie vormen de primaire informatiebronnen voor deze analyse.

\subsection{Reverse engineering van de CA Gen-applicatie}

In de tweede fase wordt een systematische reverse-engineering uitgevoerd op de geselecteerde CA Gen-applicatie. In deze fase worden alle functionaliteiten en afhankelijkheden in kaart gebracht, eventueel aangevuld met visuele ondersteuning. Daarnaast worden de belangrijkste valkuilen van CA Gen geanalyseerd en beschreven.

\subsection{Ontwerp van het migratieframework}

Op basis van de inzichten uit de probleemanalyse en de reverse-engineeringfase wordt een high-level migratieframework ontworpen. Dit framework beschrijft de opeenvolgende fasen van een migratietraject. Het stappenplan zal bestaan uit tekst, figuren en veelvoorkomende fouten, met als doel zoveel mogelijk verwarring te vermijden.

\subsection{Validatie via casestudy}

Het framework zal eerst getest worden door IT-professionals binnen het bedrijf of het duidelijk, leesbaar en effectief toepasselijk is. Ten tweede zal de correctheid van het framework getest worden aan de hand van een proof-of-concept. Als de gemigreerde code, gebruik makende van het framework, functioneel hetzelfde werkt als het origineel, wordt het gezien als een succes.

\subsection{Evaluatie en reflectie}

Tot slot wordt het migratieframework toegepast op andere aspecten van dezelfde applicatie en gecontroleerd of alles samen nog werkt. Een secundair doel is om te kijken of het framework ook werkt voor Cobol.
%---------- Verwachte resultaten ----------------------------------------------
\section{Verwacht resultaat, conclusie}%
\label{sec:verwachte_resultaten}

Een duidelijk en makkelijk te volgen stappenplan dat het migreren van CA Gen applictaties naar andere code evident maakt.



\printbibliography[heading=bibintoc]

\end{document}